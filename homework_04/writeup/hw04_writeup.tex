
\documentclass[11pt]{exam} % https://www.ctan.org/pkg/exam?lang=en

\usepackage[lmargin=1.in,rmargin=1.in,tmargin=1.in,bmargin=1in]{geometry}
\usepackage{setspace}
\usepackage[pdftex]{graphicx}
\usepackage{titling}
\usepackage[
	pdfauthor={Brian Weinstein},
	pdftitle={Homework 4},
	bookmarks=true,
	colorlinks=true,
	linkcolor=blue,
	urlcolor=blue,
	citecolor=blue,
	pdftex,
	linktocpage=true
	]{hyperref}
\usepackage[textsize=tiny]{todonotes}
\usepackage{float}
\setlength\parindent{0pt}
\usepackage{lipsum}
\usepackage{amsmath}
\usepackage{caption}


\qformat{\textbf{Problem \thequestion: \thequestiontitle}\quad \hfill}


\pagestyle{headandfoot}
\runningheadrule
\firstpageheader{}{}{}
\runningheader{\theauthor}{\thetitle}{\thedate}
\firstpagefooter{}{\thepage}{}
\runningfooter{}{\thepage}{}


\usepackage{xcolor}
\usepackage{adjustbox}
\usepackage{verbatim}
\definecolor{shadecolor}{rgb}{.9, .9, .9}

\newenvironment{code}%
   {\par\noindent\adjustbox{margin=1ex,bgcolor=shadecolor,margin=0ex \medskipamount}\bgroup\minipage\linewidth\verbatim}%
   {\endverbatim\endminipage\egroup}

\newenvironment{codeSmall}%
   {\par\noindent\adjustbox{margin=1ex,bgcolor=shadecolor,margin=0ex \medskipamount}\bgroup\minipage\linewidth\verbatim\footnotesize}%
   {\endverbatim\endminipage\egroup}

\newcommand{\ramsey}{\href{http://www.statisticalsleuth.com/}{Ramsey }}



\begin{document}


\title{STAT W4201 001, Homework 4}
\author{Brian Weinstein (bmw2148)}
\date{Feb 24, 2016}
\maketitle

Code is attached here and also posted at \href{https://github.com/BrianWeinstein/advanced-data-analysis}{https://github.com/BrianWeinstein/advanced-data-analysis}. Where relevant, code snippets and output are are included in-line.

\begin{questions}



\titledquestion{\ramsey 5.23}

The data provides overwhelming evidence that the mean oxygen isotopic composition in the 12 bone samples are different (a p-value of $9.7\times 10^{-7}$ from a one-way analysis of variance (ANOVA) F-test).

The ANOVA table testing for a difference in mean oxygen isotopic composition is shown below, and a boxplot of oxygen composition for each bone is shown in Figure \ref{fig:1}.

\begin{center}
  \begin{tabular}{ l c c c c c}
    Source of Variation & Sum of Squares & d.f. & Mean Square & F-Statistic & p-Value \\ \hline \hline
    Between Groups & 6.0675 & 11 & 0.55159 & 7.4268 & $9.73\times 10^{-7}$ \\
    Within Groups & 2.9708 & 40 & 0.07427 & & \\ \hline
    Total & 9.0383 & 51 & & &
  \end{tabular}
\end{center}
\quad \\

\begin{figure}[!h]
	\centering
	\captionsetup{width=0.8\textwidth}
	\includegraphics[width=4.25in]{1.png}
	\caption{Oxygen Isotopic Composition (per mil deviations from SMOW) for twelve bones of a single Tyrannosaurus rex specimen.}
	\label{fig:1}
\end{figure}

\pagebreak

\titledquestion{\ramsey 5.25}

\begin{parts}
\setlength{\parindent}{1em}

\part \textit{How strong is the evidence that at least one of the five population distributions (corresponding to the different years of education) is different from the others?}

Figure \ref{fig:2a} shows the distribution of income for 5 different ``years of education'' groupings. The boxplots show (1) the presence of severe outliers, and (2) that the group standard deviations increase as the years of education increases.

\begin{figure}[!h]
	\centering
	\captionsetup{width=0.8\textwidth}
	\includegraphics[width=4.25in]{2a.png}
	\caption{Income vs Years of Education for 2,584 observations among 5 ``Years of Education'' groupings.}
	\label{fig:2a}
\end{figure}

Examining the dataset, we also see that the group sample sizes are different.

\begin{codeSmall}
> # check group sample sizes and standard deviations
> incomeEduData %>%
+   group_by(Educ) %>%
+   summarize(numObs=n(), mean=mean(Income2005),
+             median=median(Income2005), stdev=sd(Income2005))
Source: local data frame [5 x 5]

    Educ numObs     mean median    stdev
  (fctr)  (int)    (dbl)  (dbl)    (dbl)
1    <12    136 28301.45  23500 21021.90
2     12   1020 36864.90  31000 29369.73
3  13-15    648 44875.96  38000 33913.54
4     16    406 69996.97  56500 64256.80
5    >16    374 76855.46  60500 65428.29
\end{codeSmall}

The F-tests are not robust to a lack of equal standard deviations and are not resistant to severe outliers. Since the education groups with higher mean incomes also have higher spreads, this dataset is a good candidate for a log transformation.

On the log scale, there are fewer outliers (which are also less-severe) and the standard deviations are nearly equal, as shown in Figure \ref{fig:2b} and in the table below.

\begin{figure}[!h]
	\centering
	\captionsetup{width=0.8\textwidth}
	\includegraphics[width=4.25in]{2b.png}
	\caption{Log(Income) vs Years of Education for 2,584 observations among 5 ``Years of Education'' groupings.}
	\label{fig:2b}
\end{figure}

\begin{codeSmall}
> # check group sample sizes and standard deviations on log scale
> incomeEduData %>%
+   group_by(Educ) %>%
+   summarize(numObs=n(), mean=mean(LogIncome2005),
+             median=median(LogIncome2005), stdev=sd(LogIncome2005))
Source: local data frame [5 x 5]

    Educ numObs     mean   median     stdev
  (fctr)  (int)    (dbl)    (dbl)     (dbl)
1    <12    136  9.89934 10.06453 0.9988809
2     12   1020 10.22721 10.34174 0.8539854
3  13-15    648 10.39121 10.54534 0.9288173
4     16    406 10.79709 10.94196 0.9581051
5    >16    374 10.89790 11.01036 1.0665910
\end{codeSmall}

Performing the one-way ANOVA F-test on the log-transformed incomes we find overwhelming evidence that at least one of the five population distributions is different from the others. The ANOVA table is shown below.

\begin{center}
  \begin{tabular}{ l c c c c c}
    Source of Variation & Sum of Squares & d.f. & Mean Square & F-Statistic & p-Value \\ \hline \hline
    Between Groups & 217.65 & 4 & 54.413 & 62.87 & $2.2\times 10^{-16}$ \\
    Within Groups & 2232.12 & 2579 & 0.865 & & \\ \hline
    Total & 2449.774 & 2583 & & &
  \end{tabular}
\end{center}
\quad \\

Although it isn't entirely justified here (as per Display 3.6), when performing the ANOVA F-test on the dataset excluding outliers\footnote{Here, an outlier is defined as an observation more than 1.5 times the group interquartile range below the first quartile or above the third quartile.} we still find overwhelming evidence that at least one distribution is different from the others, so the results are not included here.



\part \textit{By how many dollars or by what percent does the mean or median for each of the last four categories exceed that of the next lowest category?}

The \texttt{CompareTwoEducGroups} function (see attached code for function definition) performs a two-sample t-test to test the hypothesis that the mean log income in the first specified ``Years of Education'' (YOE) group is greater than the mean log income in the second specified group. It outputs a one-sided p-value, an estimated value for the multiplicative treatment effect (in USD --- the original scale), and a 95\% confidence interval for the multiplicative treatment effect (also in USD).

\begin{subparts}
\setlength{\parindent}{1em}


\subpart \textbf{($>$16) vs (16)}

\begin{codeSmall}
> CompareTwoEducGroups(data_frame=incomeEduData, Educ_groups=c(">16", "16"))
 oneSidedPVal      estimate confInt_lower confInt_upper 
   0.08238005    1.10607335    0.95934487    1.27524345 
> CompareTwoEducGroups(data_frame=incomeEduDataExclOutliers, Educ_groups=c(">16", "16"))
 oneSidedPVal      estimate confInt_lower confInt_upper 
 3.968804e-05  1.224599e+00  1.107807e+00  1.353704e+00 
\end{codeSmall}

The data provides little evidence that the $>$16 YOE population earns a higher income than the 16 YOE population (one-sided p-value $0.08238$; two-sample t-test). A 95\% confidence interval for the number of times by which the $>$16 YOE income exceeds the 16 YEO income is $0.95934$ to $1.27524$ times. 

When excluding outliers, however, (see Figure \ref{fig:2c}) the data provides convincing evidence that the $>$16 YEO population earns a higher income than the 16 YOE population (one-sided p-value $3.97 \times 10^{-5}$; two-sample t-test). Income is estimated to be $1.22460$ times greater for the those with $>$16 YOE compared to those with 16 YOE, with a 95\% confidence interval of $1.10781$ to $1.35370$ times (i.e., an estimated 22\% increase; 95\% CI from 11\% to 35\%).

\begin{figure}[!h]
	\centering
	\captionsetup{width=0.8\textwidth}
	\includegraphics[width=4.25in]{2c.png}
	\caption{Log(Income) vs Years of Education for 2,432 observations (i.e., excluding 152 outliers) among 5 ``Years of Education'' groupings.}
	\label{fig:2c}
\end{figure}



\subpart \textbf{(16) vs (13--15)}

\begin{codeSmall}
> CompareTwoEducGroups(data_frame=incomeEduData, Educ_groups=c("16", "13-15"))
 oneSidedPVal      estimate confInt_lower confInt_upper 
 7.649007e-12  1.500615e+00  1.335230e+00  1.686486e+00 
\end{codeSmall}

The data provides convincing evidence that the 16 YEO population earns a higher income than the 13--15 YOE population (one-sided p-value $7.65 \times 10^{-12}$; two-sample t-test). Income is estimated to be $1.50062$ times greater for the those with 16 YOE compared to those with 13--15 YOE, with a 95\% confidence interval of $1.33523$ to $1.68649$ times (i.e., an estimated 50\% increase; 95\% CI from 34\% to 69\%).


\subpart \textbf{(13--15) vs (12)}

\begin{codeSmall}
> CompareTwoEducGroups(data_frame=incomeEduData, Educ_groups=c("13-15", "12"))
 oneSidedPVal      estimate confInt_lower confInt_upper 
 0.0001140482  1.1782093370  1.0799490482  1.2854099405 
\end{codeSmall}

The data provides convincing evidence that the 13--15 YEO population earns a higher income than the 12 YOE population (one-sided p-value $0.00011$; two-sample t-test). Income is estimated to be $1.17821$ times greater for the those with 13--15 YOE compared to those with 12 YOE, with a 95\% confidence interval of $1.07995$ to $1.28541$ times (i.e., an estimated 17\% increase; 95\% CI from 8.0\% to 29\%).


\subpart \textbf{(12) vs ($<$12)}


\begin{codeSmall}
> CompareTwoEducGroups(data_frame=incomeEduData, Educ_groups=c("12", "<12"))
 oneSidedPVal      estimate confInt_lower confInt_upper 
 0.0000204658  1.3880147741  1.1872749406  1.6226949185 
\end{codeSmall}

The data provides convincing evidence that the 12 YEO population earns a higher income than the $<$12 YOE population (one-sided p-value $0.00002$; two-sample t-test). Income is estimated to be $1.38801$ times greater for the those with 12 YOE compared to those with $<$12 YOE, with a 95\% confidence interval of $1.18727$ to $1.62269$ times (i.e., an estimated 39\% increase; 95\% CI from 19\% to 62\%).

\end{subparts}


\end{parts}


\titledquestion{\ramsey 6.12}



\titledquestion{\ramsey 6.15}



\titledquestion{\ramsey 6.16}



\titledquestion{\ramsey 6.23}




\end{questions}

\listoftodos

\end{document}