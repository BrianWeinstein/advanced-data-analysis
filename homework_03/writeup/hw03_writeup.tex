
\documentclass[11pt]{exam} % https://www.ctan.org/pkg/exam?lang=en

\usepackage[lmargin=1.in,rmargin=1.in,tmargin=1.in,bmargin=1in]{geometry}
\usepackage{setspace}
\usepackage[pdftex]{graphicx}
\usepackage{titling}
\usepackage[
	pdfauthor={Brian Weinstein},
	pdftitle={Homework 1},
	bookmarks=true,
	colorlinks=true,
	linkcolor=blue,
	urlcolor=blue,
	citecolor=blue,
	pdftex,
	linktocpage=true
	]{hyperref}
\usepackage[textsize=tiny]{todonotes}
\usepackage{float}
\setlength\parindent{0pt}
\usepackage{lipsum}
\usepackage{amsmath}


\qformat{\textbf{Problem \thequestion: \thequestiontitle}\quad \hfill}


\pagestyle{headandfoot}
\runningheadrule
\firstpageheader{}{}{}
\runningheader{\theauthor}{\thetitle}{\thedate}
\firstpagefooter{}{\thepage}{}
\runningfooter{}{\thepage}{}


\usepackage{xcolor}
\usepackage{adjustbox}
\usepackage{verbatim}
\definecolor{shadecolor}{rgb}{.9, .9, .9}

\newenvironment{code}%
   {\par\noindent\adjustbox{margin=1ex,bgcolor=shadecolor,margin=0ex \medskipamount}\bgroup\minipage\linewidth\verbatim}%
   {\endverbatim\endminipage\egroup}

\newenvironment{codeSmall}%
   {\par\noindent\adjustbox{margin=1ex,bgcolor=shadecolor,margin=0ex \medskipamount}\bgroup\minipage\linewidth\verbatim\footnotesize}%
   {\endverbatim\endminipage\egroup}

\newcommand{\ramsey}{\href{http://www.statisticalsleuth.com/}{Ramsey }}



\begin{document}


\title{STAT W4201 001, Homework 3}
\author{Brian Weinstein (bmw2148)}
\date{Feb 17, 2016}
\maketitle

Code is attached here and also posted at \href{https://github.com/BrianWeinstein/advanced-data-analysis}{https://github.com/BrianWeinstein/advanced-data-analysis}. Where relevant, code snippets and output are are included in-line.

\begin{questions}


\titledquestion{\ramsey 4.30} % Problem 1


\titledquestion{\ramsey 4.32} % Problem 2



\titledquestion{\ramsey 5.19} % Problem 3



\titledquestion{} % Problem 4
\textit{Consider the Bumpus’s data in Chapter 2, compute the power of the two-sided two sample t-test of size 0.05 (i.e., reject the null hypothesis if the absolute value the t-statistic is greater than or equal to 2), under the alternative that $\mu_x - \mu_y = \overline{x} - \overline{y} = 0.01$ and $\sigma = s_p = 0.0214$.}


\titledquestion{} % Problem 5
\textit{Show that the two-sided two sample t-test is equivalent to the anova F-test, if the number of groups is two.}

For $I=2$ groups, the F-statistic is given by
$$ \text{F-statistic} =  \frac{SS_B / \left[ (n-1) - (n-I) \right]}{SS_W / (n-I)}, $$
where $n_1$ and $n_2$ are the sizes of samples 1 and 2, respectively, $n=n_1+n_2$ is the total sample size, $SS_B$ is the ``between groups'' sum of squared residuals, and $SS_W$ is the ``within groups'' sum of squared residuals.

Simplifying, we find
$$\text{F-statistic} =  \frac{SS_B / (I-1) }{SS_W / (n-I)} = \frac{SS_B / (2-1) }{SS_W / (n-2)} = \frac{SS_B / 1 }{SS_W / (n-2)}.$$

If the observations from group 1 are $\sim \text{N}(\mu_1, \sigma^2)$ and the observations from group 2 are $\sim \text{N}(\mu_2, \sigma^2)$, we know that
$$\text{F-statistic} \sim \text{F}_{1, n-2} \text{ , which is equivalent to }t^2_{n-2}.$$
i.e., an F distribution with a numerator degrees of freedom of 1 and a denominator degrees of freedom of $n-2$ is equivalent to the square of a t distribution with $n-2$ degrees of freedom. \todo{Finish problem 5}



\titledquestion{} % Problem 6
\textit{Consider $X_1,\ldots,X_{10}$ are i.i.d. N($0,\sigma^2$), $Y_1,\ldots,Y_{10}$ are i.i.d. N($\mu,\sigma^2$) and hypothesis testing:}
\begin{align*}
H_0 &: \mu = 0\\
H_A &: \mu \neq 0.
\end{align*}
\textit{Compute the power of a two sided two sample t-test of size 0.05 when $\sigma^2 = 1$ and $\mu =$ 0.1, 0.5, 1, and 2. Plot the power as a function of $\mu$. Then, increase the sample size in each group to 20 and draw the power function in the same plot as that of the sample size 10.}



\titledquestion{} % Problem 7
\textit{Under the setting of the previous problem, show that, under the null hypothesis, the p-value follows the uniform distribution on the interval [0, 1] and perform simulations to confirm it.}




\end{questions}

\listoftodos

\end{document}