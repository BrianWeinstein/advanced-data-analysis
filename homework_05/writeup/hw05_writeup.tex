
\documentclass[11pt]{exam} % https://www.ctan.org/pkg/exam?lang=en

\usepackage[lmargin=1.in,rmargin=1.in,tmargin=1.in,bmargin=1in]{geometry}
\usepackage{setspace}
\usepackage[pdftex]{graphicx}
\usepackage{titling}
\usepackage[
	pdfauthor={Brian Weinstein},
	pdftitle={Homework 1},
	bookmarks=true,
	colorlinks=true,
	linkcolor=blue,
	urlcolor=blue,
	citecolor=blue,
	pdftex,
	linktocpage=true
	]{hyperref}
\usepackage[textsize=tiny]{todonotes}
\usepackage{float}
\setlength\parindent{0pt}
\usepackage{lipsum}
\usepackage{amsmath}
\usepackage{caption}


\qformat{\textbf{Problem \thequestion: \thequestiontitle}\quad \hfill}


\pagestyle{headandfoot}
\runningheadrule
\firstpageheader{}{}{}
\runningheader{\theauthor}{\thetitle}{\thedate}
\firstpagefooter{}{\thepage}{}
\runningfooter{}{\thepage}{}


\usepackage{xcolor}
\usepackage{adjustbox}
\usepackage{verbatim}
\definecolor{shadecolor}{rgb}{.9, .9, .9}

\newenvironment{code}%
   {\par\noindent\adjustbox{margin=1ex,bgcolor=shadecolor,margin=0ex \medskipamount}\bgroup\minipage\linewidth\verbatim}%
   {\endverbatim\endminipage\egroup}

\newenvironment{codeSmall}%
   {\par\noindent\adjustbox{margin=1ex,bgcolor=shadecolor,margin=0ex \medskipamount}\bgroup\minipage\linewidth\verbatim\footnotesize}%
   {\endverbatim\endminipage\egroup}

\newcommand{\ramsey}{\href{http://www.statisticalsleuth.com/}{Ramsey }}



\begin{document}


\title{STAT W4201 001, Homework 5}
\author{Brian Weinstein (bmw2148)}
\date{March 2, 2016}
\maketitle

Code is attached here and also posted at \href{https://github.com/BrianWeinstein/advanced-data-analysis}{https://github.com/BrianWeinstein/advanced-data-analysis}. Where relevant, code snippets and output are are included in-line.

\begin{questions}


\titledquestion{\ramsey 7.18}

\begin{parts}

\part \textit{Find the standard error of prediction for the prediction of pH at 5 hours after slaughter.}

The calculations in Display 7.12 give us $\hat{\beta}_0=6.9836$, $\hat{\beta}_1=-0.7257$, $\hat{\sigma}=0.08226$, $n=10$, $\overline{X}=1.190$, $s_X^2=0.6344$.

Therefore, the standard error of prediction for the pH at 5 hours is
\begin{align*}
\text{SE}\left[ \text{Pred}\{ Y | X_0=\log(5)=1.609438  \} \right] &= \hat{\sigma} \sqrt{ 1 + \frac{1}{n} + \frac{(X_0 - \overline{X})^2}{(n-1)s_X^2} } \\
&= (0.08226) \sqrt{ 1 + \frac{1}{10} + \frac{(1.609438 - 1.190)^2}{9 \cdot 0.6344} } \\
&= 0.0875
\end{align*}


\part \textit{Construct a 95\% prediction interval at 5 hours after slaughter.}

The prediction of pH level at 5 hours is,
\begin{align*}
{Pred}\{ Y | X_0=\log(5)=1.609438  \}  &=  \beta_0 + \beta_1 \cdot \log(5) \\
&= 6.9836 - 0.7257 \cdot 1.609438 \\
&= 5.8156.
\end{align*}

A 95\% prediction confidence interval at 5 hours is given by
\begin{gather*}
5.8156 \pm t_{8}(0.975) \cdot \text{SE}\left[ \text{Pred}\{ Y | X_0=\log(5) \} \right] \\
5.8156 \pm 2.3060 \cdot 0.0875 \\
5.8156 \pm 0.2017 \\
\Rightarrow [5.6139, 6.0173].
\end{gather*}


\end{parts}


\titledquestion{\ramsey 7.24}

\begin{parts}

\part \textit{With a statistical computer package and the data in the file ex0724, obtain the least squares fits to the four simple regressions, individually, to confirm the estimates and standard errors presented in Display 7.17.}

Confirming the estimates and standard errors from Display 7.17:

\begin{subparts}

\subpart Denmark
\begin{codeSmall}
> lmDenmark <- lm(formula=Denmark~Year, data=birthData)
> summary(lmDenmark)$coefficients
                  Estimate   Std. Error   t value     Pr(>|t|)
(Intercept)  0.59872329381 0.0408047207 14.672893 2.395722e-18
Year        -0.00004288538 0.0000206916 -2.072598 4.423828e-02
\end{codeSmall}$

\subpart The Netherlands
\begin{codeSmall}
> lmNetherlands <- lm(formula=Netherlands~Year, data=birthData)
> summary(lmNetherlands)$coefficients
                  Estimate   Std. Error   t value     Pr(>|t|)
(Intercept)  0.67239837505 0.0279195810 24.083398 1.365923e-26
Year        -0.00008084321 0.0000141577 -5.710196 9.636921e-07
\end{codeSmall}$

\subpart Canada
\begin{codeSmall}
> lmCanada <- lm(formula=Canada~Year, data=birthData)
> summary(lmCanada)$coefficients
                 Estimate    Std. Error   t value     Pr(>|t|)
(Intercept)  0.7337857143 0.05480068278 13.390083 3.983523e-11
Year        -0.0001111688 0.00002767698 -4.016653 7.375947e-04
\end{codeSmall}$

\subpart United States
\begin{codeSmall}
> lmUsa <- lm(formula=USA~Year, data=birthData)
> summary(lmUsa)$coefficients
                  Estimate     Std. Error   t value     Pr(>|t|)
(Intercept)  0.62008571429 0.018598766807 33.340152 2.523643e-18
Year        -0.00005428571 0.000009393273 -5.779212 1.439109e-05
\end{codeSmall}$

\end{subparts}

\part \textit{Obtain the t-statistic for the test that the slopes of the regressions are zero, for each of the four countries. Is there evidence that the proportion of male births is truly declining?}

The t-statistics and associated two-sided p-values are computed and shown in the output of part (a). For the \texttt{Year} variable:

\begin{subparts}

\subpart Denmark

The t-statistic is $-2.0726$, with a one-sided p-value of $0.0221$. The data provides moderate, but not convincing evidence, that the proportion of male births is truly declining in Denmark.


\subpart The Netherlands

The t-statistic is $-5.7102$, with a one-sided p-value of $0.000000482$. The data provides overwhelming evidence that the proportion of male births is truly declining in the Netherlands.


\subpart Canada

The t-statistic is $-4.0167$, with a one-sided p-value of $0.000369$. The data provides convincing evidence that the proportion of male births is truly declining in Canada.


\subpart United States

The t-statistic is $-5.7792$, with a one-sided p-value of $0.00000720$. The data provides overwhelming evidence that the proportion of male births is truly declining in the United States.

\end{subparts}


\part \textit{Explain why the United States can have the largest of the four t-statistics (in absolute value) even though its slope is only the third largest (in absolute value).}

In the hypothesis that the slopes of the regressions are zero, the t-statistic is defined as $Estimate/\text{SE}(Estimate)$. So even though the slope is only the third largest (in absolute value), the standard error on the estimate is small enough to make the $Estimate/\text{SE}(Estimate)$ ratio largest for the USA.

\part \textit{Explain why the standard error of the estimated slope is smaller for the United States than for Canada, even though the sample size is the same.}

The standard error of the estimated slope is given by
$$\text{SE}(\hat{\beta}_1) = \hat{\sigma} \sqrt{\frac{1}{(n-1)s_X^2}},$$
where $s_X^2$ is the sample variance of the $X$'s.

For the US and Canada, $n$ and $s_X^2$ are the same (it's the same sample size, using the same subset of years), but $\hat{\sigma}$ is different.

$$\hat{\sigma} = \sqrt{\frac{\text{RSS}_j}{\text{Degrees of freedom}}} = \sqrt{\frac{\text{RSS}_j}{n-2}} = \sqrt{\frac{\text{RSS}_j}{19}},$$

where $\text{RSS}_j$ is the sum of squared residuals for group j.

For the US, the RSS is
\begin{codeSmall}
> sum((lmUsa$residuals)^2)
[1] 0.000001290857
\end{codeSmall}$

and for Canada, the RSS is
\begin{codeSmall}
> sum((lmCanada$residuals)^2)
[1] 0.00001120681
\end{codeSmall}$

Since the RSS for the US model is an order of magnitude smaller than the RSS for the Canada model (and since all other parameters in $\text{SE}(\hat{\beta}_1)$ are identical) the standard error for the estimated slope is smaller for the US.

\part \textit{Can you think of any reason why the standard deviations about the regression line might be different for the four countries? (Hint: The proportion of males is a kind of average, i.e., the average number of births that are male.)}

There's no reason to think that the standard deviations about the regression lines would be the same for the four countries. These are samples from four different populations, and there's no reason to think their sampling distributions will have identical spreads.


\end{parts}



\titledquestion{\ramsey 7.28}



\titledquestion{\ramsey 8.17}



\titledquestion{\ramsey 8.20}



\titledquestion{\ramsey 9.12}





\end{questions}

%\listoftodos

\end{document}